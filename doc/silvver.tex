\documentclass[a4paper,10pt]{article}
\usepackage[brazil]{babel}
\usepackage[utf8]{inputenc}
\usepackage[T1]{fontenc}
\usepackage{graphicx}
\usepackage{listings}
\usepackage{numprint}
\usepackage{tikz}
\usepackage[colorlinks=true,linkcolor=blue,citecolor=blue,urlcolor=blue]{hyperref}
\title {Silvver \\ Sistema de Localização por Visão VeRLab}
\author{Renato Florentino Garcia}

\usetikzlibrary{calc,3d}

\include{lua}
\lstset{numbers=left, numberstyle=\tiny, stepnumber=2, numbersep=5pt,
  basicstyle=\small\tt, language=lua}

\begin{document}

\maketitle
\newpage

\tableofcontents
\newpage
\section{Introdução}
Esta documentação está desatualizada

O Silvver é um sistema que foi desenvolvido para facilitar a obtenção da
localização de alvos no ambiente. Ele possibilita que um usuário consiga de
forma transparente as localização de alvos obtidos através de vários toolkits
de localização, como por exemplo o ARToolKitPlus.

O Silvver foi construído com o objetivo de ser um sistema de localização
modular e multi-plataforma. A modularidade possibilita uma maior
escalabilidade ao sistema, cada um de seus módulos pode ser executado em uma
máquina diferente, dividindo entre elas o processamento do todo. Além disso,
os diferentes computadores podem ser espalhados espacialmente, tornando mais
fácil o uso de várias câmeras para cobrir uma grande área.

O sistema é dividido em três módulos: silvver-cameras, silvver-server e
libsilvver-client. Cada um desses módulos é um programa independente dos
demais; o silvver-cameras e o silvver-server são programas executáveis, e a
libsilvver-client é um biblioteca. Cada programa pode ser executado ou em
computadores diferentes ou na mesma máquina, e a comunicação entre eles se dá
por meio sockets.

\subsection{Silvver-cameras}

O silvver-cameras é o módulo responsável por controlar as câmeras e processar
as imagens capturadas por elas. Cada instância desse módulo pode controlar
quantas câmeras forem necessárias, e esse módulo pode ser instanciado tantas
vezes, e em quantos computadores forem necessários.

O silvver-cameras é configurado por um arquivo que descreve a cena que que
será controlada por ele. Este arquivo de configuração é um script escrito na
linguagem Lua\cite{lua}, e nele estão todas as câmeras que serão usadas e seus
parâmetros, bem como todos os tipos de alvos e suas configurações.

\subsection{Silvver-server}

O silvver-server deve ser instanciado uma única vez em todo o sistema. Ele tem
a tarefa de receber todos os alvos que foram localizados pelos vários
silvver-cameras, verificar se algum deles foi visto por mais de uma câmera ao
mesmo tempo e eliminar a duplicidade, e enviar as localizações para os
clientes interessados.

Ao ser iniciado o silvver-server passa imediatamente a ouvir uma porta TCP/IP
pré-estabelecida. A entidade ligada a esta porta é chamada recepcionista, e
todos os pedidos para adicionar ou terminar a conexão com uma câmera, e
adicionar ou retirar um cliente é recebido primeiramente por ela.


\subsection{Libsilvver-client}

A libsilvver-client é uma biblioteca C++ que fornece uma API para o acesso às
localizações de um dado alvo.

\section{Instalação}

A compilação de cada um dos módulos é feita individualmente, por serem eles
programas independentes. A única exigência quanto a bibliotecas é que a versão
da Boost\cite{boost} usada na compilação seja a mesma em todos eles, pois como
os objetos que serão enviados pela rede são serializados usando a Boost
Serialization, e versões antigas dessa biblioteca não necessariamente são
compatíveis com as serializações feitas por versões mais novas, problemas de
comunicação podem surgir.

Na raiz do projeto existem três diretórios: common, libsilvver\_client,
silvver\_cameras e silvver\_server. Em cada um dos diretórios
libsilvver\_client, silvver\_cameras e silvver\_server estão os fontes dos
respectivos módulos; já o diretório common contém os fontes que são
compartilhados entre os módulos.

Todo o projeto Silvver usa o SCons\cite{scons} como ferramenta de
compilação. Portanto, em cada uma das pastas dos módulos há um arquivo
\emph{SConstruct}, que é o script que verificas quais bibliotecas estão
instaladas no sistema, e faz a compilação e a instalação. As opções de
compilação de cada módulo podem ser vistas usando o comando \texttt{scons -h}
no mesmo diretório onde está o arquivo \emph{SConstruct}.

O processo para compilar e instalar os três módulos é excencialmente o mesmo,
por isso um mesmo exemplo pode ser seguido para todos eles, com pequenas
alterações se necessário. Dado um ambiente linux com a Boost versão 1.39
instalada em um diretório não padrão, com as bibliotecas em
\emph{/usr/local/bin} e os cabeçalhos em
\emph{/usr/local/include/boost-1\_39}. O Silvver pode ser instalado dentro do
diretório \emph{/home/renatofg/silvver} com o seguinte comando: \texttt{scons
  install prefix=/home/renatofg/silvver/usr lib\_boost\_suffix=-gcc43-mt-1\_39
  LIBPATH=/usr/local/lib/ CPPPATH=/usr/local/include/boost-1\_39/}.

\section{Exemplo de uso}

O primeiro módulo que dever ser executado é o silvver-server. A opção mais
importante que este ele possui é -p ou --receptionist-port, que permite
selecionar em qual porta TCP/IP o recepcionista estará ouvindo. Caso não seja
especificado, a porta 12000 será usada.

Após executado 

\section{O Arquivo de Configuração}

O arquivo de configuração do silvver-cameras é um script escrito na
linguagem de progração Lua\cite{lua}. Ao inciar o silvver-cameras criará um
ambiente lua para executar o \emph{script}, e após a execução lerá a variável
scene, que deverá ser uma tabela contento toda as estruturas necessárias.
A única variável de interesse ao silvver-cameras é a scene, todas as outras
variáveis que por ventura estejam no ambiente serão ignoradas.  Um exemplo
completo é dado abaixo:

\begin{lstlisting}[frame=lines]
require("cameraId")
require("cameraConstructors")

scene = {
    cameras = {
        Dragonfly{
            uid = dragonfly2uid(5110432),

            focal_length = {531.20605, 531.45104},
            principal_point = {359.2638, 278.30819},
            radial_coef = {-0.43208, 0.236, 0},
            tangential_coef = {-0.00054004, 0.003849},

            translation_vector = {358.43, -2.3797, 921.3742},
            rotation_matrix = {
                0.9960135, 0.0050411, -0.0890,
                0.0013316, -0.99913, -0.04166,
                    -0.08919, 0.04137, -0.99515
            },

            shutter = 500,
            gain = 800,
            white_balance = {50, 5},
        },
    },

    artp_targets = {

        pattern_width = 150,

        {
            pattern_file = 'data/4x4patt/4x4_1.patt',
            uid = 1,
        },
    },
}
\end{lstlisting}

As linhas 1 e 2 incluem dois fontes com funções que implementam algumas
facilidades. Cada câmera possui um número que a identifica univocamente (UID),
e dependendo da biblioteca utilizada, como a libdc1394, esse número difere do
UID dado pelo fabricante. Em camedaId estão funções que traduzem os
identificadores dados pelo fabricante da câmera para o UID esperado pela
biblioteca. Esse é o caso de dragonfly2uid, que recebe o UID no formato dado
nas câmeras dragonfly e retorna um UID no formato esperado pela libdc1394.

Cada câmera possui diversas opções de configuração, e para que não seja
necessário especificar cada um desses parâmetros.

\section{Calibração das Câmeras}

O processo de calibração de uma câmera consiste de se procurar os melhores
parâmetros para um modelo teórico que a descreva através de equações
matemáticas, sendo que estes parâmetros podem ser divididos entre intrínsecos
e extrínsecos. Os parâmetros intrínsecos definem as características opticas,
geométricas e digitais da câmera. No Silvver, os parâmetros intrínsecos de
interesse são: distância focal, ponto principal, três parâmetros de distorção
radial e dois parâmetros de distorção tangencial. Os parâmetros extrínsecos
por sua vez descrevem como a câmera está posicionada no mundo, através de um
vetor de deslocamento e uma matriz de rotação é dada a pose da câmera em
relação a uma origem no mundo.

Dois scripts para a calibração de câmeras fazem parte do projeto Silvver, são
eles \emph{silvverIntCalib} e \emph{silvverExtCalib}, que são usados para
determinar os parâmetros intrínsecos e extrínsecos respectivamente.

\subsection{SilvverIntCalib}

\begin{figure}
  \centering
  \includegraphics[width=0.5\columnwidth]{figures/checkerboard}
  \caption{Exemplo de imagem usada para a calibração intrínseca de uma
    câmera.}
  \label{fig:checkerboard}
\end{figure}

A calibração intrínseca usando o script \emph{silvverIntCalib} é feita por
meio de imagens de um tabuleiro xadrez capturadas pela câmera em questão, como
a imagem da figura \ref{fig:checkerboard}. O primeiro passo portanto para a
calibração intrínseca é salvar uma sequência de imagens em que seja possível
ver completamente um tabuleiro. Uma vez que se tenha as imagens, basta
executar o script passando-as como argumento.

Assumindo-se que se esteja no mesmo diretório que as imagens, todas elas com
extensão .ppm; e o tabuleiro seja o mesmo da figura \ref{fig:checkerboard},
com 10$\times$7 casas, e cada casa medindo \numprint[mm]{25}, a calibração
poderia ser obtida com o seguinte comando: \texttt{silvverIntCalib 10 7 0.025
  *.ppm}. O silvverIntCalib também aceita algumas opções, que podem ser vistas
pelo comando \texttt{silvverIntCalib -h}.

\subsection{SilvverExtCalib}

\begin{figure}[h]
  \centering
  \begin{tikzpicture}
    \draw (0,0,0) -- (1,0,0);
    \draw (0,0,0) -- (0,1,0);
    \draw (0,0,0) -- (0,0,1);
  \end{tikzpicture}
  \caption{oi}
  \label{fig:silvverExtCalib}
\end{figure}

O problema da calibração extrínseca resolvido pelo silvverExtCalib é definido
pelo seguinte enunciado: em um espaço tridimensional, seja uma pose
$\mathrm{P_1}$ e uma câmera C. São conhecidas as coordenadas (x, y, z, e a
matriz de rotação) da pose $\mathrm{P_1}$ em relação a uma origem O
($\mathrm{P_1^O}$), e em relação à camera C ($\mathrm{P_1^C}$). Deseja-se
obter a pose da camera C em relação à origem O ($\mathrm{P_C^O}$). Ou seja,
tendo-se $\mathrm{P_1^O}$ e $\mathrm{P_1^C}$ deseja-se conhecer
$\mathrm{P_C^O}$, como está ilustrado na figura \ref{fig:silvverExtCalib}. O
que o silvverExtCalib faz é essa transformação vetorial.


Uma forma prática se obter $\mathrm{P_1^O}$ e $\mathrm{P_1^C}$, que são
pré-requisitos do processo de calibração, é utilizando um alvo para o qual o
Silvver forneça uma pose 6D. Vamos assumir que a pose desse alvo seja
$\mathrm{P_1}$. Caso se deseje que $\mathrm{P_1}$ seja a origem do mundo, para
determinar $\mathrm{P_1^O}$ basta fazer $(x,y,z) = (0,0,0)$ e a matriz de
rotação igual à matriz identidade. Caso $\mathrm{P_1}$ não esteja na origem do
mundo, sua pose $\mathrm{P_1^O}$ pode ser dada usando o Silvver e uma segunda
câmera já calibrada.

Para encontrar $\mathrm{P_1^C}$ basta configurar o silvver-cameras tendo a
câmera C na origem do mundo, e obter a localização de $\mathrm{P_1}$. Para
isso, o arquivo de configuração lua teria o seguinte trecho de código na
tabela referente à camera C:
\begin{lstlisting}[numbers=none]
    translation_vector = {0, 0, 0},
    rotation_matrix = {
        1, 0, 0,
        0, 1, 0,
        0, 0, 1},
\end{lstlisting}

O formato de entrada de $\mathrm{P_1^O}$ e $\mathrm{P_1^C}$ esperado pelo
silvverExtCalib é um arquivo de texto para cada uma, dentro do qual está sua
pose. Se $\mathrm{P_1^O}$ e/ou $\mathrm{P_1^C}$ foram obtidas usando o
Silvver, possivelmente seu arquivo de texto correspondente é formado por
várias linhas, onde cada uma delas é a localização de $\mathrm{P_1}$ em um
dado instante de tempo. Neste caso o silvverExtCalib tomará a média desses
valores como a pose que será utilizada.

Tendo-se $\mathrm{P_1^O}$ em um arquivo de texto nomeado \emph{p1o.txt}, e
$\mathrm{P_1^C}$ em um nomeado \emph{p1c.txt}, para encontrar $\mathrm{P_C^O}$
basta executar o seguinte comando estando-se no mesmo mesmo diretório em que
estão os arquivos \emph{p1o.txt} e \emph{p1c.txt}: \texttt{silvverExtCalib
  p1o.txt p1c.txt}.

\section{Desenvolvimento}


\bibliography{silvver}
\bibliographystyle{plain}

\end{document}
