\documentclass[a4paper,10pt]{article}
\usepackage[brazil]{babel}
\usepackage[utf8]{inputenc}
\usepackage[T1]{fontenc}
\usepackage{graphicx}
\title {Silvver \\ Sistema de Localização por Visão VERLab}
\author{Renato Garcia}
\date{}


\begin{document}
\maketitle

\section{Introdução}
Esta documentação está desatualizada

O Silvver é um sistema descentralizado de localização por visão. Ele
efetua localização de marcos visuais pré-determinados a partir de
imagens digitais.


\section{O programa}
O Silvver foi construído com o objetivo de ser um sistema de
localização modular e multi-plataforma. A modularidade possibilita uma
maior escalabilidade ao sistema, cada um de seus módulos pode ser
executados em uma máquina diferente, dividindo entre elas o
processamento do todo. Além disso, os diferentes computadores podem
ser distribuídos procurando uma posição que minimize

é constituído por três módulos: libsilvver-cliente, silvver-servidor e
silvver-câmeras. Silvver-câmeras é o módulo responsável por controlar
A comunicação entre estes módulos é feita através de sockets,
portanto, cada um deles pode ser executado em um computador diferente,
desde que estejam ligados em rede ou tenham acesso à internet.

\section{Exemplo de uso}
\subsection{silvver-câmera}
Silvver-câmera é o módulo responsável por controlar as câmeras e
processar as imagens obtidas. Cada computador deve executar no máximo
uma única instância deste programa, e antes de iniciar a execução, um
silvver-servidor deve estar obrigatoriamente sendo executado em alguma
máquina cujo endereço IP seja conhecido. Caso hajam múltiplas câmeras
conectadas em um mesmo computador, esta única instância manipulará
todo o conjunto.

\subsubsection{cameras.xml}
Toda as opções de execução e configuração são fornecidas ao programa
através de arquivo xml cujo nome deverá ser necessariamente
cameras.xml. Este arquivo descreve as propriedades de cada câmera
disponível, e se elas serão ou não utilizadas. Abaixo está ilustrado
um aquivo de configuração cameras.xml.


\begin{verbatim}
<camera ligada="sim">
   <modelo> PGR </modelo>
   <serial> 5111119 </serial>
   <diretorio> E:/renato/cam19/ </diretorio>
   <resolucao> 640 480 </resolucao>
   <frequencia> 30 </frequencia>
   <distancia_focal> 530.09343 529.65974 </distancia_focal>
   <ponto_principal> 341.69169 219.77290 </ponto_principal>
   <distorcao_radial>
       -0.41522   0.16578   0.00092   -0.0016 0.0 0.0
   </distorcao_radial>
   <alfa_c> 0 </alfa_c>
   <matrizH>
       4.654740  0.212210 -5845.671545
       0.031842 -4.724545 -2008.903768
      -0.040346  0.012023 -1347.414986
   </matrizH>
</camera>
\end{verbatim}

Onde as etiquetas utilizadas tem o seguinte significado:
\begin{itemize}
\item modelo: Utilizado internamente por silvver-cameras para
  identificar a classe que será utilizada para controlar a captura de
  imagens da câmera. É necessário que o modelo da camêra tenha uma
  classe correspondente implementada no silvver-cameras.
\item serial: Uma identificação única para a câmera
\end{itemize}



% \input{silvver_cliente_h}




\end{document}
